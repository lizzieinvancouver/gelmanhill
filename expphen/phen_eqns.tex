\documentclass[12pt,a4paper]{article}
\usepackage[top=1.00in, bottom=1.0in, left=1in, right=1in]{geometry}
\usepackage{amsmath}

\begin{document}

\noindent To understand how experimental warming and precipitation treatments affect soil moisture ($y_{i}$) I want to fit a hierachical model with 3 levels: one level for site, represented here by $j$; one level for year, represented here by $k$; and one level for day of year, represented here by $l$. $\beta_{1}$ is the warming treatment and $\beta_{2}$ is the precipitation treatment.  I am also interested in their interactive effects. Each observation is $i$, and partial pooling on the intercept ($\alpha$) only:

% no commas

\begin{equation}
y_{i}=\alpha_{j[i]1}+\alpha_{k[i]2}+\alpha_{l[i]3}+\beta_{1}X_{i1}+\beta_{2}X_{i2}+\beta_{1}\beta_{2}X_{i1}X_{i2}+\epsilon_{i}
\end{equation}
The multilevel parts of the model are:
\begin{equation}
\alpha_{j1} \sim N(\mu_{\alpha_1}, \sigma^{2}_{\alpha_1})
\end{equation}

\begin{equation}
\alpha_{k2} \sim N(\mu_{\alpha_2}, \sigma^{2}_{\alpha_2})
\end{equation}

\begin{equation}
\alpha_{l3} \sim N(\mu_{\alpha_3}, \sigma^{2}_{\alpha_3})
\end{equation}



\noindent To understand how soil moisture  affects GDDcrit($y_{i}$) (the accumulated growing degree days on the day of a phenoogical event such as bud burst), I want to fit a hierachical model with 2 levels: one level for site, represented here by $j$; and one level for species, represented here by $k$;  $\beta_{1}$ is soil moisture (averaged over the whole year- not sure if this is best).  Each observation is $i$, and partial pooling could be on the intercept ($\alpha$) only or on the slope and intercept. I'm not sure which is best....shown here for pooling on both slope and intercept for species, and intercept-only for site:

% no commas

\begin{equation}
y_{i}=\alpha_{j[i]1}+\alpha_{k[i]2}+\beta_{1}X_{k[i]1}+\epsilon_{i}
\end{equation}
The multilevel parts of the model are:
\begin{equation}
\alpha_{j1} \sim N(\mu_{\alpha_1}, \sigma^{2}_{\alpha_1})
\end{equation}

\begin{equation}
\alpha_{k2} \sim N(\mu_{\alpha_2}, \sigma^{2}_{\alpha_2})
\end{equation}

\begin{equation}
\beta_{k1} \sim N(\mu_{\beta_1}, \sigma^{2}_{\beta_1})
\end{equation}


\end{document}

